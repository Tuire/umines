\begin{abstract}
In this literature review we introduce exploratory search and cover three techniques that help users in succeeding in their exploratory search tasks efficiently. Exploratory search is an information retrieval strategy that combines learning and investigation activities. We explain the concept of exploratory search and what distinguishes it from other information retrieval strategies. Exploratory searchers can be supported in their tasks by means of user modeling. After shortly summarizing user modeling we present three practical techniques that combine user modeling with exploratory search. 
\textit{Faceted search} is a method where structured metadata is used to provide the user with an overview of the results and clickable categories. 
\textit{Relevance feedback} provides the user an option to rate the usefulness of the result. This information is used in ranking the search results next time. 
\textit{Query term suggestion} displays choosable query suggestions as the user is typing their query.
\end{abstract}
