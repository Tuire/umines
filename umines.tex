\documentclass{sigchi}

% Use this command to override the default ACM copyright statement (e.g. for preprints). 
% Consult the conference website for the camera-ready copyright statement.
\toappear{
First draft
% Submitted for review.
}

% Arabic page numbers for submission. 
% Remove this line to eliminate page numbers for the camera ready copy
\pagenumbering{arabic}


% Load basic packages
\usepackage{balance}  % to better equalize the last page
\usepackage{graphics} % for EPS, load graphicx instead
\usepackage{times}    % comment if you want LaTeX's default font
\usepackage{url}      % llt: nicely formatted URLs

% llt: Define a global style for URLs, rather that the default one
\makeatletter
\def\url@leostyle{%
  \@ifundefined{selectfont}{\def\UrlFont{\sf}}{\def\UrlFont{\small\bf\ttfamily}}}
\makeatother
\urlstyle{leo}


% To make various LaTeX processors do the right thing with page size.
\def\pprw{8.5in}
\def\pprh{11in}
\special{papersize=\pprw,\pprh}
\setlength{\paperwidth}{\pprw}
\setlength{\paperheight}{\pprh}
\setlength{\pdfpagewidth}{\pprw}
\setlength{\pdfpageheight}{\pprh}

% Make sure hyperref comes last of your loaded packages, 
% to give it a fighting chance of not being over-written, 
% since its job is to redefine many LaTeX commands.
\usepackage[pdftex]{hyperref}
\hypersetup{
pdftitle={User Modeling in Exploratory Search},
pdfauthor={LaTeX},
pdfkeywords={SIGCHI, proceedings, archival format},
bookmarksnumbered,
pdfstartview={FitH},
colorlinks,
citecolor=black,
filecolor=black,
linkcolor=black,
urlcolor=black,
breaklinks=true,
}

% create a shortcut to typeset table headings
\newcommand\tabhead[1]{\small\textbf{#1}}


% End of preamble. Here it comes the document.
\begin{document}

\title{User Modeling in Exploratory Search}

\def\tktl{Department of Computer Science\\University of Helsinki}
\def\tktladdr{P.O. Box 68 (Gustaf H\"allstr\"omin katu 2b)\\
FI-00014 UNIVERSITY OF HELSINKI\\
FINLAND}

\numberofauthors{2}
\author{
  \alignauthor Ilkka Kiistala\\
    \affaddr{\tktl}\\
    % \affaddr{\tktladdr}\\
    \email{ilkka.kiistala@helsinki.fi}
  \alignauthor Tuire Peurala\\
    \affaddr{\tktl}\\
    % \affaddr{\tktladdr}\\
    \email{tuire.peurala@helsinki.fi}
}

\maketitle

\begin{abstract}
This is abstract.
\end{abstract}

\keywords{
	Exploratory Search; Information Retrieval; User Modeling.
}

\category{H.5.m.}{Information Interfaces and Presentation (e.g. HCI)}{Miscellaneous}


\terms{
	Human Factors; Design; Measurement. 
}

\section{Introduction}
This is the introduction.


\section{User modeling}
\label{sec:usermodeling}
Shortish explanation of user modeling key concepts. 
\cite{rich99}, \cite{fischer01}

\subsection{Sterotypes}
Modeling stereotypes. 
% HCI reference needed
\cite{dillon96}, \cite{pu02}

\subsection{How to Collect and Analyze User Information}
% Pazzani puhuu "user profileista"
\cite{pazzani97}, \cite{white10}

\subsection{Personalization}
Individualization of user models, Adaptive/Adaptable User Interfaces, intelligent user interfaces
\cite{bunt04}, \cite{findlater04}, \cite{brusi96}

\section {Exploratory search is a subtopic of information retrieval}

\subsection{Information retrieval}
There are many goals in information retrieval and exploratory search is one of them.
\cite{hearst02}, \cite{kuhlt91}

\subsection{Exploratory Search}
Introduction to exploratory search.
\cite{march06}, \cite{white09}, \cite{tvaro11}

\section{User modeling in exploratory search}
How has user modeling been used in supporting exploratory search, example cases? What challenges have emerged? 
\cite{oconnor10}, \cite{sugi04}, \cite{white07}, \cite{kules09}

\subsection{Evaluation of Exploratory Search Systems}
What are the challenges in evaluating Exploratory Search Systems?
\cite{whitemm08}, \cite{kules08}

\section{Conclusion}
Here are the conclusions.

\section{Who added what references?}

\cite{rich99} Tuire \\
\cite{fischer01} Tuire \\
\cite{dillon96} Tuire \\
\cite{pu02} Tuire \\
\cite{pazzani97} Ilkka \\
\cite{white10} Ilkka \\
\cite{bunt04} Ilkka \\
\cite{findlater04} Ilkka \\
\cite{brusi96} Ilkka \\
\cite{hearst02} Tuire \\
\cite{kuhlt91} Tuire \\
\cite{march06} Tuire \\
\cite{white09} Ilkka \\
\cite{tvaro11} Ilkka \\
\cite{oconnor10} Tuire \\
\cite{sugi04} Tuire \\
\cite{white07} Ilkka \\
\cite{kules09} Tuire \\
\cite{whitemm08} Ilkka \\
\cite{kules08} Ilkka \\


\subsection{Reference count}

\begin{tabular}{ l c c }
Author & Tuire & Ilkka\\
\hline
References added & 10 & 10\\
\end{tabular}




% \nocite{*} %tämä listaa kaikki viitteet luetteloon vaikka niitä ei olisi vielä viitattu



% Balancing columns in a ref list is a bit of a pain because you
% either use a hack like flushend or balance, or manually insert
% a column break.  http://www.tex.ac.uk/cgi-bin/texfaq2html?label=balance
% multicols doesn't work because we're already in two-column mode,
% and flushend isn't awesome, so I choose balance.  See this
% for more info: http://cs.brown.edu/system/software/latex/doc/balance.pdf
%
% Note that in a perfect world balance wants to be in the first
% column of the last page.
%
% If balance doesn't work for you, you can remove that and
% hard-code a column break into the bbl file right before you
% submit:
%
% http://stackoverflow.com/questions/2149854/how-to-manually-equalize-columns-
% in-an-ieee-paper-if-using-bibtex
%
% Or, just remove \balance and give up on balancing the last page.
%
\balance

% If you want to use smaller typesetting for the reference list,
% uncomment the following line:
% \small
\bibliographystyle{acm-sigchi}
\bibliography{umines}
\end{document}
