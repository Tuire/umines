\section{Discussion}

In the modern day people use online search applications widely. Actually search applications are the second most frequently used online applications [kirja]. Online environment seems to be perfect for both searching  fast lookup information and doing more sophisticated searching with goals like investigating, forming overall understanding of a given subject or trying to learn uncharted territories. People's online lives may include anything from finding the perfect companion or seeing Psy's next tremendous viral hit video to learning the basics of physics.  Built on the idea of hyperlinking the internet is practically made for exploring. Exploratory search tasks have become part of everyday life, maybe also because the internet has applications that support it better than the traditional library, books and conversation approach.

Exploratory searchers can be supported in their tasks by means of user modeling, the goal being to accurately model the users' information needs. This task is unfortunately a very difficult one. Indeed, it is hard for users themselves to precisely describe what their information need eventually even is. 

The user interface of an exploratory search system should be designed to fulfill the needs of most of its users. 
More information on what works and doesn't work can usually be collected from system evaluations.
However, evaluating exploratory search systems is difficult, because users have different starting positions [].
Their knowledge of the domain varies, they are interested in different aspects of the topic and they have previously encountered different information.
The solutions that aid exploratory search usually include methods to recommend alternative search paths or methods to suggest sources of related information. 

Faceted search has been found to be a useful means of supporting exploratory search. It supports the refining of the query and enables browsing through the formed categories thus enriching the understanding of the knowledge domain. Additionally, the faceted search helps the users to refine or re-evaluate their information needs. Still, even though faceted search interfaces make the search more efficient and the users explore the results more broadly than without facets, they don't always prefer it. 

Another supporting technique that we came upon was query term suggestion. It saves the user from typing as the system suggests combinations of query terms based on the characters or words the user has already typed in the search form. A study suggests that as much as 15 percent of search queries return no results because of misspelled query terms. By clicking a suggested search query, the search gets underway with correctly spelled words and with a certainty that some results will be returned, since the suggested query has successfully been used before.

Most search systems rely only on the query the user has submitted when constructing a user model, but more direct ways to gather information exist, too. Asking user to rate the result based on their usefulness provides the system with direct and precise evidence. Relevance feedback has been found effective in enhancing search accuracy [], but this method requires system to analyse given feedback and apply it into the ranking the search results. In practise, the extra effort the user needs to make seems to be too much for the users.

In our opinion using user models in supporting exploratory search has some payoffs. The user might feel insecure if the application forms search results for reasons the user does not understand, depriving the user's control of the system. User modeling also has privacy implications. Who stores the information that has been gathered from the user's interactions? How securely it is stored and who controls the purposes it is used for? For example, Google collects information about the user's actions and uses them to construct user model for marketing purposes. The privacy of using the data is questionable and they provide no means of preventing them from collecting one's interaction data. 

The methods we encountered during our literature review were familiar to us as major search engines have incorporated them in their user interfaces granted that we didn't know their names, let alone their theoretical background. It is obvious to us that the solutions described in our paper are useful to many exploratory searchers although not all of the implementations we tried were adequately intuitive or mature enough in their visualization.

